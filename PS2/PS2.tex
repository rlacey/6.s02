%%%%%%%%%%%%%%%%%%%%%%%%%%%%%%%%%%%%%%%%%
% University/School Laboratory Report
% LaTeX Template
% Version 3.0 (4/2/13)
%
% This template has been downloaded from:
% http://www.LaTeXTemplates.com
%
% Original author:
% Linux and Unix Users Group at Virginia Tech Wiki 
% (https://vtluug.org/wiki/Example_LaTeX_chem_lab_report)
%
% License:
% CC BY-NC-SA 3.0 (http://creativecommons.org/licenses/by-nc-sa/3.0/)
%
%%%%%%%%%%%%%%%%%%%%%%%%%%%%%%%%%%%%%%%%%

%----------------------------------------------------------------------------------------
%	PACKAGES AND DOCUMENT CONFIGURATIONS
%----------------------------------------------------------------------------------------

\documentclass{article}

\usepackage[version=3]{mhchem} % Package for chemical equation typesetting
\usepackage{siunitx} % Provides the \SI{}{} command for typesetting SI units

\usepackage{graphicx}
\usepackage{caption}
\usepackage{subcaption}
\usepackage{cancel}

\usepackage{float}

\usepackage[T1]{fontenc} % allow small bold caps

\setlength\parindent{0pt} % Removes all indentation from paragraphs

\renewcommand{\labelenumi}{\alph{enumi}.} % Make numbering in the enumerate environment by letter rather than number (e.g. section 6)

\usepackage[margin=1in]{geometry}

\usepackage{amssymb}

%\usepackage{times} % Uncomment to use the Times New Roman font

%----------------------------------------------------------------------------------------
%	Title
%----------------------------------------------------------------------------------------

\begin{document}
\pagenumbering{gobble}

\title{6.s02: EECS II - From A Medical Perspective}
\author{
  Ryan Lacey <rlacey@mit.edu>\\
  \footnotesize \texttt{Collaborator(s): Jorge Perez}
}
        
\maketitle
        


\begin{enumerate}

\item[1.]
	\begin{itemize}
	\item
		Area of probability distribution sums up to one, so area of triangle is one.\\
		
		$1 = \frac{1}{2} \left(b \cdot h \right)= \frac{1}{2} \left( 4 \cdot a \right)$\\
		
		$a = \frac{1}{2}$\\

	\item Mean of distribution\\

		$\mu_x = \displaystyle\int_{-\infty}^{\infty} xf_x(x)dx$\\
		
		$\mu_x = \displaystyle\int_{0}^{2} x \left(\frac{x}{4}\right)dx + \displaystyle\int_{2}^{4} x \left(1- \frac{x}{4}\right)dx$\\

		$\mu_x = \frac{2}{3} + \frac{4}{3} = 2$\\

	\item Variance of distribution\\

		$\sigma^2_x = \displaystyle\int_{-\infty}^{\infty} \left(x - \mu_x\right)^2f_x(x)dx$\\
		
		$\sigma^2_x = \displaystyle\int_{0}^{2} \left(x - 2\right)^2 \left(\frac{x}{4}\right)dx + \displaystyle\int_{2}^{4} \left(x - 2\right)^2 \left(1- \frac{x}{4}\right)dx$\\
		
		$\sigma^2_x = \frac{1}{3} + \frac{1}{3} = \frac{2}{3}$\\
	\end{itemize}

\newpage

\item[2.]
	\begin{enumerate}
	\item[(a)]4
	\item[(b)]3
	\item[(c)]5
	\item[(d)]2
	\item[(e)]6
	\item[(f)]1
	\item[(g)]7
	\item[(h)]The patient moves toward the $E$ region. For the top $C$ the patient is measured at a higher level than they have and will move to an even lower level. For the bottom $C$ region the patient is measured at a lower level than they have and will move to an even higher level.
	\end{enumerate}

\newpage

\item[3.]
	\begin{enumerate}
	\item[(a)]
			$\sin\left(\frac{6\pi}{7}\left(n + N\right)\right) = \sin\left(\frac{6\pi}{7}n + \frac{6\pi}{7}N\right)$\\
			
			$\sin\left(\frac{6\pi}{7}n + \frac{6\pi}{7}N\right) = \sin\left(\frac{6\pi}{7}n + 2\pi k\right)$\\
			
			$\therefore \frac{6\pi}{7}N = 2\pi k$\\
			
			$N = 7$ (satisfied by $k=3$)\\

	\item[(b)]
			$\sin\left(\frac{6\pi}{7}\left(n + N\right) + 1\right) = \sin\left(\left(\frac{6\pi}{7}n + \frac{6\pi}{7}N\right) + 1\right)$\\
			
			The one is just a shift and does not affect the period, so we can ignore it. Then the equation becomes the same as the above.\\
			
			$\sin\left(\frac{6\pi}{7}n + \frac{6\pi}{7}N\right) = \sin\left(\frac{6\pi}{7}n + 2\pi k\right)$\\
			
			$\therefore \frac{6\pi}{7}N = 2\pi k$\\
			
			$N = 7$ (satisfied by $k=3$)\\

	\item[(c)]
			$\cos\left(\frac{6\left(n + N\right)}{7} - \pi\right) = \cos\left(\left(\frac{6n}{7} + \frac{6N}{7}\right) - \pi\right)$\\
			
			As before, we can ignore the shift term by dropping $\pi$ from our concerns.\\
			
			$\cos\left(\frac{6n}{7} + \frac{6N}{7}\right) = \cos\left(\frac{6n}{7} + 2\pi k\right)$\\
			
			$\therefore \frac{6N}{7} = 2\pi k$\\
			
			$N = \frac{7}{3}\pi k$\\
			
			No integer value of $k$ satisfies, so it is not a DT periodic signal.\\

	\item[(d)]
			$\cos\left(\frac{\pi}{2}\left(n + N\right)\right)\cos\left(\frac{\pi}{4}\left(n + N\right)\right) = \cos\left(\frac{\pi}{2}n + \frac{\pi}{2}N\right)\cos\left(\frac{\pi}{4}n + \frac{\pi}{4}N\right)$\\
			
			$\cos\left(\frac{\pi}{2}n + \frac{\pi}{2}N\right)\cos\left(\frac{\pi}{4}n + \frac{\pi}{4}N\right) = \cos\left(\frac{\pi}{2}n + 2\pi k_1\right)\cos\left(\frac{\pi}{4}n + 2\pi k_2\right)$\\
			
			$\therefore \frac{\pi}{2}N = 2\pi k_1$ and $\frac{\pi}{4}N = 2\pi k_2$\\
			
			$N = 8$ (satisfied by $k_1=2$ and $k_2=1$)\\
	\end{enumerate}

\newpage

\item[4.]
	\begin{enumerate}
	\item[(a)]
			$X\left[k\right] = \dfrac{1}{N} \displaystyle\sum_{n=0}^{N-1} x\left[n\right]e^{-j\frac{2\pi}{N}kn}$\\
			
			$X\left[k\right] = \dfrac{1}{6} \displaystyle\sum_{n=0}^{5} \left(1 + \cos\left(\frac{2\pi}{6}n\right)\right)e^{-j\frac{2\pi}{6}kn}$\\
			
			Solved with WolframAlpha\\
			
			$X\left[k\right] = \frac{1}{4}e^{\left( -5j \frac{\pi}{3}k\right)} + \frac{1}{12}e^{\left( -4j \frac{\pi}{3}k\right)} + \frac{1}{12}e^{\left( -2j \frac{\pi}{3}k\right)} + \frac{1}{4}e^{\left(\frac{\pi}{3}\left(-k\right)\right)} + \frac{1}{3}$\\
			
	\item[(b)]
			$X\left[k\right] = \dfrac{1}{8} \displaystyle\sum_{n=0}^{3}e^{-j\frac{2\pi}{8}kn} - \dfrac{1}{8} \displaystyle\sum_{n=4}^{7}e^{-j\frac{2\pi}{8}kn}$\\
	\end{enumerate}

\newpage

\item[5.]
	$x\left[n\right] = \displaystyle\sum_{k=-\frac{N}{2}}^{\frac{N}{2}-1} X\left[k\right]e^{j\frac{2\pi}{N}kn}$
	
	$x\left[n\right] = \displaystyle\sum_{k=-4}^{3} X\left[k\right]e^{j\frac{2\pi}{7}kn}$\\
	
	$x\left[n\right] = -\frac{1}{2j}e^{j\frac{-4\pi}{7}n} + e^{j\frac{-2\pi}{7}n} + e^{j\frac{2\pi}{7}n} + \frac{1}{2j}e^{j\frac{4\pi}{7}n}$\\
	
	$x\left[n\right] = \frac{1}{2j}\left(-e^{j\frac{-4\pi}{7}n} + e^{j\frac{4\pi}{7}n}\right) + 1\left(e^{j\frac{-2\pi}{7}n} + e^{j\frac{2\pi}{7}n}\right)$\\
	
	$x\left[n\right] = \sin\left(\frac{4\pi}{7}n\right) + 2\cos\left(\frac{2\pi}{7}n\right)$

\end{enumerate}

\end{document}