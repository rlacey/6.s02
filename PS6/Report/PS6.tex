%%%%%%%%%%%%%%%%%%%%%%%%%%%%%%%%%%%%%%%%%
% University/School Laboratory Report
% LaTeX Template
% Version 3.0 (4/2/13)
%
% This template has been downloaded from:
% http://www.LaTeXTemplates.com
%
% Original author:
% Linux and Unix Users Group at Virginia Tech Wiki 
% (https://vtluug.org/wiki/Example_LaTeX_chem_lab_report)
%
% License:
% CC BY-NC-SA 3.0 (http://creativecommons.org/licenses/by-nc-sa/3.0/)
%
%%%%%%%%%%%%%%%%%%%%%%%%%%%%%%%%%%%%%%%%%

%----------------------------------------------------------------------------------------
%	PACKAGES AND DOCUMENT CONFIGURATIONS
%----------------------------------------------------------------------------------------

\documentclass{article}

\usepackage[version=3]{mhchem} % Package for chemical equation typesetting
\usepackage{siunitx} % Provides the \SI{}{} command for typesetting SI units

\usepackage{graphicx}
\usepackage{caption}
\usepackage{subcaption}
\usepackage{cancel}

\usepackage{float}

\usepackage[T1]{fontenc} % allow small bold caps

\usepackage{listings}
\usepackage{color}

\definecolor{dkgreen}{rgb}{0,0.6,0}
\definecolor{gray}{rgb}{0.5,0.5,0.5}
\definecolor{mauve}{rgb}{0.58,0,0.82}

\lstset{frame=tb,
  language=Matlab,
  aboveskip=2mm,
  belowskip=2mm,
  showstringspaces=false,
  columns=flexible,
  basicstyle={\small\ttfamily},
  numbers=none,
  numberstyle=\tiny\color{gray},
  keywordstyle=\color{blue},
  commentstyle=\color{dkgreen},
  stringstyle=\color{mauve},
  breaklines=true,
  breakatwhitespace=true
  tabsize=2
}

\setlength\parindent{0pt} % Removes all indentation from paragraphs

\renewcommand{\labelenumi}{\alph{enumi}.} % Make numbering in the enumerate environment by letter rather than number (e.g. section 6)

\usepackage[margin=1in]{geometry}

\usepackage{amssymb}


%\usepackage{times} % Uncomment to use the Times New Roman font

%----------------------------------------------------------------------------------------
%	Title
%----------------------------------------------------------------------------------------

\begin{document}
\pagenumbering{gobble}

\title{6.s02: EECS II - From A Medical Perspective}
\author{
  Ryan Lacey <rlacey@mit.edu>\\
  \footnotesize \texttt{Collaborator(s): Jorge Perez}
}
        
\maketitle
        


\begin{enumerate}
\item[1.]
	\begin{enumerate}
	\item[(a)]
		Average heart rate: 87.7 bpm\\
		Standard deviation of RR interval: 0.0021\\
\begin{lstlisting}   
function [ times, rrt ] = ecg2rr( ecg, minheight, mindistance )
    t = linspace(0, 30+(5/60), length(ecg));
    [pks, locs] = findpeaks(ecg, 'MINPEAKHEIGHT', minheight, 'MINPEAKDISTANCE', mindistance);
    times = diff(t(locs));
    rrt = t(locs(2:end));
end

sum(times.^-1)/length(times) %bpm
std(rrt) %std
\end{lstlisting}

\bigskip

	\item[(b)]
		K=2\\
		Centroids: 0.0105, 0.0137\\
		Heart rates: 95.12, 73.23\\
\begin{lstlisting}   
[IDX, C] = kmeans(times, 2, 'Start', [min(times); max(times)])
C %centroids
C.^-1 %heart rates
\end{lstlisting}

\bigskip

	\item[(c)]
		K=3\\
		Centroids: 0.0101, 0.0127, 0.0182\\
		Heart rates: 99.22, 78.81, 54.84\\
		Clusters: <1, 1146> <2, 1307> <3, 106>\\
\begin{lstlisting}   
[IDX, C] = kmeans(times, 3, 'Start', [min(times); mean(times); max(times)])
C %centroids
C.^-1 %heart rates
\end{lstlisting}
	\end{enumerate}

\newpage

\item[2.]
	\begin{enumerate}
	\item[(a)]
		Longest beat: 547 elements
	\end{enumerate}
\end{enumerate}

\end{document}